\documentclass[a4paper, 12pt]{article}
\usepackage[utf8x]{inputenc}
\usepackage{cmap}
\usepackage[english, russian]{babel}
\usepackage{indentfirst}
\usepackage[left=20mm, top=20mm, right=20mm, bottom=20mm]{geometry}
\usepackage{tikz}
\usepackage{float}
\usepackage{amsmath, amsfonts, amssymb}
\usepackage{graphicx}
\usepackage{fancybox, fancyhdr}
\usepackage{hyperref}
\usepackage{listings}
\usepackage{caption}
\usepackage{subcaption}
\usepackage{xcolor}
\usepackage{paralist}
\pagestyle{fancy}
\fancyhf{}
\fancyhead[L]{Лабораторная работа №1}
\fancyhead[R]{Линейные системы автоматического управления}
\fancyfoot[C]{\thepage}
\graphicspath{{images/}}
\usetikzlibrary{patterns}
\definecolor{LightGray}{gray}{0.95}
\definecolor{LightGray2}{gray}{0.7}
\hypersetup{
    colorlinks=true,
    linkcolor=blue,
    filecolor=magenta,
    urlcolor=cyan,
    pdftitle={contents setup},
    pdfpagemode=FullScreen,
}
\setlength{\parskip}{1.5mm}
\setlength{\headheight}{15pt}
\setlength{\footskip}{15pt}
\allowdisplaybreaks
\newcommand{\frc}[2]{\raisebox{2pt}{$#1$}\big/\raisebox{-3pt}{$#2$}}

\begin{document}
    \begin{titlepage}

        \begin{center}
        \includegraphics[width=0.3\textwidth]{itmo.png} % requires itmo.png in /images folder
        \vfill
        
        Федеральное государственное автономное образовательное учреждение высшего образования
        «Национальный Исследовательский Университет ИТМО»\\
        
        \vfill
        {\large\bf ЛАБОРАТОРНАЯ РАБОТА №1}\\
        {\large\bf ПРЕДМЕТ «ЛИНЕЙНЫЕ СИСТЕМЫ АВТОМАТИЧЕСКОГО УПРАВЛЕНИЯ»}\\
        {\large\bf ТЕМА «МОДЕЛИРОВАНИЕ ЛИНЕЙНЫХ ДИНАМИЧЕСКИХ СИСТЕМ»}\\
        Вариант 4
        \vfill

        \begin{flushright}
            \begin{minipage}{.45\textwidth}
            {
                \hbox{Преподаватель: Золотаревич В. П.}
                \hbox{Студент: Румянцев А. А.}
                \hbox{Поток: ЛСАУ R22 бак 4.1.1}
                \hbox{}
                \hbox{Факультет: СУиР}
                \hbox{Группа: R3341}
            }
            \end{minipage}
        \end{flushright}
        
        \vfill
                
        Санкт-Петербург\\
        2024
        \end{center}
    \end{titlepage}
    
    \tableofcontents

    \newpage
    \section{Задание 1}
    \subsection{Условие}
    \textit{Исследование модели вход-выход:}
    \begin{compactitem}
    \item Построить схему моделирования линейной динамической системы при
    $$n=3,\ \ a_0=8,\ \ a_1=6,\ \ a_2=2,\ \ b_0=12,\ \ b_1=1,\ \ b_2=10$$
    \item Осуществить моделирование системы при двух видах входного воздействия
    $$u=1(t),\ \ u=2\sin{(t)}$$ и нулевых начальных условиях. Выводить графики
    сигналов $u(t)$ и $y(t)$. Продолжительность интервала наблюдения выбрать самостоятельно.
    \item Осуществить моделирование свободного движения системы, т.е. с нулевым входным
    воздействием и ненулевыми начальными условиями
    $$y(0)=1,\ \ \dot{y}(0)=0.1,\ \ \ddot{y}(0)=-0.1$$ Выводить графики $y(t)$.
    \end{compactitem}


    \subsection{Выполнение}
    $$y^{(3)}+2y^{(2)}+6y^{(1)}+8y=10u^{(2)}+u^{(1)}+12u$$
    $$p^3y+2p^2y+6py+8y=10p^2u+pu+12u$$
    $$p^3y=-2p^2y-6py-8y+10p^2u+pu+12u$$
    $$y=-\dfrac{2}{p}y-\dfrac{6}{p^2}y-\dfrac{8}{p^3}y+\dfrac{10}{p}u+\dfrac{1}{p^2}u+\dfrac{12}{p^3}u$$
    $$y=\dfrac{1}{p}\left(10u-2y\right)+\dfrac{1}{p^2}\left(u-6y\right)+\dfrac{1}{p^3}\left(12u-8y\right)$$
   
    $$z_1=y\Rightarrow z_1(0)=y(0)=1$$
    $$\dot{y}=\dot{z}_1=z_2+10u-2y\Rightarrow z_2=\dot{y}-10u+2y$$
    $$z_2(0)=\dot{y}(0)-10u(0)+2y(0)=0.1-0+2=2.1$$
    $$\dot{z}_2=z_3+u-6y\Rightarrow z_3=\dot{z_2}-u+6y$$
    $$\dot{z}_2=\ddot{y}-10\dot{u}+2\dot{y}\Rightarrow z_3=\ddot{y}-10\dot{u}+2\dot{y}-u+6y$$
    $$z_3=\ddot{y}(0)-10\dot{u}(0)+2\dot{y}(0)-u(0)+6y(0)=-0.1-0+2\cdot0.1-0+6\cdot1=6.1$$


    \section{Задание 2}
    \subsection{Условие}
    \textit{Исследование модели вход-состояние-выход:}
    \begin{compactitem}
    \item Построить схему моделирования линейной динамической системы при
    $$n=2,\ \
    A=
    \begin{bmatrix}
        0 & -4\\
        1 & -1
    \end{bmatrix},\ \
    B=
    \begin{bmatrix}
        0.5\\
        0.25
    \end{bmatrix},\ \
    C^T=
    \begin{bmatrix}
        0\\
        8
    \end{bmatrix}$$
    \item Осуществить моделирование линейной динамической системы при двух видах входного воздействия
    $$u=1(t),\ \ u=2\sin{(t)}$$ Выводить графики
    сигналов $u(t)$ и $y(t)$. Начальное значение вектора состояния нулевое.
    \item Осуществить моделирование свободного движения системы с начальными условиями
    $$x_1(0)=-0.5,\ \ x_2(0)=0.13$$ Выводить графики $y(t)$.
    \end{compactitem}


    \subsection{Выполнение}
    $$
    \begin{cases}
        \dot{x}=Ax+Bu\\
        y=Cx
    \end{cases}
    \Rightarrow
    \begin{cases}
    \dot{x}_1=0x_1-4x_2+0.5u\\
    \dot{x}_2=1x_1-1x_2+0.25u\\
    y=0x_1+8x_2
    \end{cases}
    \Rightarrow
    \begin{cases}
        \dot{x}_1=-4x_2+0.5u\\
        \dot{x}_2=x_1-x_2+0.25u\\
        y=8x_2
    \end{cases}
    $$
\end{document}