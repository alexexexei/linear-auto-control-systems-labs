\documentclass[a4paper, 12pt]{article}
\usepackage[utf8x]{inputenc}
\usepackage{cmap}
\usepackage[english, russian]{babel}
\usepackage{indentfirst}
\usepackage[left=20mm, top=20mm, right=20mm, bottom=20mm]{geometry}
\usepackage{tikz}
\usepackage{float}
\usepackage{amsmath, amsfonts, amssymb}
\usepackage{graphicx}
\usepackage{fancybox, fancyhdr}
\usepackage{hyperref}
\usepackage{listings}
\usepackage{caption}
\usepackage{subcaption}
\usepackage{xcolor}
\usepackage{paralist}
\pagestyle{fancy}
\fancyhf{}
\fancyhead[L]{Лабораторная работа №6}
\fancyhead[R]{Линейные системы автоматического управления}
\fancyfoot[C]{\thepage}
\graphicspath{{images/}}
\usetikzlibrary{patterns}
\definecolor{LightGray}{gray}{0.95}
\definecolor{LightGray2}{gray}{0.7}
\hypersetup{
    colorlinks=true,
    linkcolor=blue,
    filecolor=magenta,
    urlcolor=cyan,
    pdftitle={contents setup},
    pdfpagemode=FullScreen,
}
\setlength{\parskip}{1.5mm}
\setlength{\headheight}{15pt}
\setlength{\footskip}{15pt}
\allowdisplaybreaks

\begin{document}
    \begin{titlepage}

        \begin{center}
        \includegraphics[width=0.3\textwidth]{itmo.png} % requires itmo.png in /images folder
        \vfill
        
        Федеральное государственное автономное образовательное учреждение высшего образования
        «Национальный Исследовательский Университет ИТМО»\\
        
        \vfill
        {\large\bf ЛАБОРАТОРНАЯ РАБОТА №6}\\
        {\large\bf ПРЕДМЕТ «ЛИНЕЙНЫЕ СИСТЕМЫ АВТОМАТИЧЕСКОГО УПРАВЛЕНИЯ»}\\
        {\large\bf ТЕМА «АНАЛИЗ ВЛИЯНИЯ НУЛЕЙ И ПОЛЮСОВ ПЕРЕДАТОЧНОЙ ФУНКЦИИ НА ДИНАМИЧЕСКИЕ СВОЙСТВА»}\\
        Вариант 4
        \vfill

        \begin{flushright}
            \begin{minipage}{.45\textwidth}
            {
                \hbox{Преподаватель: Золотаревич В. П.}
                \hbox{Студент: Румянцев А. А.}
                \hbox{Поток: ЛСАУ R22 бак 4.1.1}
                \hbox{}
                \hbox{Факультет: СУиР}
                \hbox{Группа: R3341}
            }
            \end{minipage}
        \end{flushright}
        
        \vfill
                
        Санкт-Петербург\\
        2024
        \end{center}
    \end{titlepage}
    
    \tableofcontents

    \newpage
    \section{Цель работы}
    Изучить связь характера переходной характеристики,
    динамических свойств системы с размещением на комплексной
    плоскости нулей и полюсов.


    \section{Задание 1}
    По заданным в табл. 6.3 значениям постоянных n t k, определите
    параметры системы (6.1) с характеристическим полиномом Баттерворта и би-
    номиальным полиномом. Для каждого случая рассчитайте корни характеристи-
    ческого полинома (6.2) и оцените время переходного процесса по формуле (6.6).
    Составьте схему моделирования системы и постройте переходные характери-
    стики, соответствующие двум типам распределения корней характеристическо-
    го уравнения.


    \section{Задание 2}
    Для каждого набора параметров b bm0 , ,… , приведенных в табл. 6.4 и
    6.5, постройте переходные характеристики системы (6.7) с коэффициентами
    a a n0 1, ,… -- и коэффициентом b, рассчитанными в п.1 для биномиального распре-
    деления корней характеристического уравнения.


    \section{Задание 3}
    Для набора параметров b bm0 , ,… и внешнего воздействия g t( ) , приве-
    денных в табл. 6.6, постройте реакцию системы (6.7) с нулевыми начальными
    условиями и коэффициентами a a n0 1,… -- , рассчитанными в п.1 для биномиаль-
    ного распределения корней характеристического уравнения. На экран монитора
    выводить графики y t g t( ), ( ) 
\end{document}